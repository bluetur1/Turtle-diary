
% \documentclass[final,3p,times]{elsarticle}
% 若要做成两列
% \documentclass[final,3p,times,twocolumn]{elsarticle}
\documentclass[11pt,a4paper,twocolumn]{article}
% 数学公式包
\usepackage{amsmath}
\usepackage{amssymb}
% 图片包
\usepackage{graphicx}
% 参考文献包
\usepackage[numbers]{natbib}
% 解决自动换行
\usepackage{tabularx}
% 三线表
\usepackage{booktabs}
% 绘制表格中更细的线
\usepackage{multirow}
% 绘制表格中竖线
\usepackage{makecell}
% 页边距包
\usepackage{geometry}
% \geometry{papersize={20cm,15cm}} 纸张大小
% \geometry{left=1cm,right=2cm,top=3cm,bottom=4cm} 页边距
% 页眉页脚
\usepackage{fancyhdr}
\pagestyle{fancy}
\lhead{} % 页眉左边
\chead{} % 页眉中间
\rhead{} % 页眉右边
\lfoot{} % 页脚左边
\cfoot{\thepage} % 页脚中间
\rfoot{} % 页脚右边
\renewcommand{\headrulewidth}{0.4pt}  % 页眉与正文之间0.4pt横线分割
\renewcommand{\headwidth}{\textwidth}
\renewcommand{\footrulewidth}{0pt}

% 行间距 字号的1.5倍
\usepackage{setspace}
\onehalfspacing
% 段间距 在原有基础上增加.4em 若要减小,改为负数即可
\addtolength{\parskip}{.4em} 

%中英文摘要
\newcommand{\enabstractname}{Abstract}
\newcommand{\cnabstractname}{摘要}
\newenvironment{enabstract}{%
\par\small
\noindent\mbox{}\hfill{\bfseries\enabstractname}\hfill\mbox{}\par
\vskip 2.5ex}{\par\vskip2.5ex}
\newenvironment{cnabstract}{%
	\par\small
	\noindent\mbox{}\hfill{\bfseries\cnabstractname}\hfill\mbox{}\par
	\vskip 2.5ex}{\par\vskip2.5ex}
% \begin{document}
% 	\maketitle
% 	\thispagestyle{empty}
% 	\newpage
% 	\setcounter{page}{1}
% 	\begin{cnabstract}
% 		中文摘要
 
% 		\noindent{\textbf{关键词:}关键词一\quad  关键词二\quad   关键词三\quad }
% %		\keywords{一\quad  二\quad   三\quad  }
% 	\end{cnabstract}
 
% 	\begin{enabstract}
% 		abstract
 
% 		\noindent{\textbf{keywords:}one ;\quad two;\quad three;}
% 	\end{enabstract}
% 	\newpage
% \end{document}

\begin{document}
% \frontmatter:关闭章节序号, 页码使用罗马数字;
% \mainmatter:开启章节序号计数,重置页码,页码使用阿拉伯数字;
% \appendix:重置章节序号计数, 章节序号使用字母,对页码没有影响;
% \backmatter:关闭章节序号,对页码没有影响。
% \begin{frontmatter}
  
% \end{frontmatter}
% 生成标题
\title{my title}
\author{heishuini}
% \address{this is my introduce}
\date{}
\maketitle    

% 生成目录
    \tableofcontents 
% 生成子标题和段落
    \section{Hello hsn1} hsn is good1
        \subsection{Hello hsn2} hsn is good2
            \subsubsection{Hello hsn3} hsn is good3
                \paragraph{para hsn1} para hsn..1
                    \subparagraph{para2 hsn} para hsn..2
        \subsection{Hello hsn2} 
            your content
            \paragraph{para hsn1} nihao 
% 换行效果 空一行:另起段落; \\ 强制换行
    \subsection{math}

% 数学公式 
Einstein 's $E=mc^2$. %行内
\[ E=mc^2. \] % 单独成行,不进行标号 
\begin{equation}
E=mc^2. % 单独成行,有标号
\end{equation}
\[ z = r\cdot e^{2\pi i}. \]
% 矩阵
  Matrix (lcr here means left, center or right for each column)
   \[
     \left[
       \begin{array}{lcr}
         a1 & b22 & c333 \\
         d444 & e555555 & f6
       \end{array}
     \right]
   \]
% \quad 是空格
   \[ \begin{pmatrix} a&b\\c&d \end{pmatrix} \quad
   \begin{bmatrix} a&b\\c&d \end{bmatrix} \quad
   \begin{Bmatrix} a&b\\c&d \end{Bmatrix} \quad
   \begin{vmatrix} a&b\\c&d \end{vmatrix} \quad
   \begin{Vmatrix} a&b\\c&d \end{Vmatrix} \]

   Marry has a little matrix $ ( \begin{smallmatrix} a&b\\c&d \end{smallmatrix} ) $.

% 等式
Equations(here \& is the symbol for aligning different rows)
\begin{align}
   a+b&=c\\
   d&=e+f+g
\end{align}
% 括号等式 aligned对齐 不编号可以带*版本
\[
   \left\{
     \begin{aligned}
       &a+b=c\\
       &d=e+f+g
     \end{aligned}
   \right.
\]
% 不对齐 multline*不编号
\begin{multline}
  x = a+b+c+{} \\
  d+e+f+g
  \end{multline}
% 分段函数
\[ y= \begin{cases}
  -x,\quad x\leq 0 \\
  x,\quad x>0
  \end{cases} \]

% 插入图片
\subsection{myFigure}
  \begin{figure}[h] % 可选h(here就这里) t(top 页顶) b(bottom 页尾) p(float page专门放专栏)
    \centering  
    \includegraphics[width=0.50in,height=0.50in]{figure1.jpg} 
    \caption{figure title}  % 插图标题
  \end{figure}

\subsection{myTable}
% 表格1
\subsubsection{Table1}
\begin{tabular}{|c|c|}
    aaa & b \\
    c & ddddd\\
  \end{tabular}

% 表格2
\subsubsection{Table2}
\begin{tabular}{c|cc}
  \Xhline{2pt}
  variable1 & \multicolumn{2}{c}{variable2} \\
  \Xcline{2-3}{0.4pt}
  r & l & s \\
  \Xhline{1pt}
  1.00 & 6.28 & 6.28 \\
  2.00 & 12.57 & 12.57 \\
  3.00 & 18.85 & 28.37 \\
  \Xhline{2pt}
  \end{tabular} 

% 表格3
\subsubsection{Table3}
\begin{tabular}{*{6}{c}}
  \toprule
  \multirow{2}*{name} & \multicolumn{2}{c}{fruit} & \multicolumn{2}{c}{vegetable} & \\
  \cmidrule(lr){2-3}\cmidrule(lr){4-5}\cmidrule(lr){6-6}\morecmidrules\cmidrule(lr){6-6}
  & apple & orange & fruit3 & fruit4 & sum \\
  \midrule
  xiaoming & 2kg & 1kg & 1.5kg & 2kg & 6.5kg \\
  xiaoming & 2kg & 1kg & 1.5kg & 2kg & 6.5kg \\
  \bottomrule
\end{tabular}

% 表格4
\subsubsection{Table4}
% \toprule命令:表格顶部的粗线。
% \midrule命令:表格中间的细分隔线。
% \bottomrule命令:表格底部的粗线。
\begin{tabular}{cccccc}
  \toprule
  num & name & sex & age & height/cm & weight/kg \\
  \midrule
  1 & zs & M & 16 & 163 & 50 \\
  2 & wh & F & 15 & 159 & 47 \\
  3 & le & M & 17 & 165 & 52 \\
  \bottomrule
\end{tabular}


% 参考文献
\subsection{my References}
One reference about watermelon \cite{gostout1992clinical}       
Another reference about watermelon \cite{rivero2001resistance}       

\bibliographystyle{plain}       
\bibliography{references}

% \end{frontmatter}
\end{document}